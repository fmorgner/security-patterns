
\documentclass{article}
\usepackage[utf8]{inputenc}
\usepackage[T1]{fontenc}
\usepackage[swissgerman]{babel}
\usepackage[left=2cm,right=1.5cm,top=1.5cm,bottom=1.5cm]{geometry}
\usepackage{amsmath, amsfonts, amssymb, graphicx,fancyhdr, hyperref, lscape, pdflscape, tabularx, ragged2e, titlesec}

\setlength\parindent{0pt} % Removes all indentation from paragraphs

\renewcommand{\labelenumi}{\alph{enumi}.} % Make numbering in the enumerate environment by letter rather than number (e.g. section 6)


\title{Authorization} % Title

\author{Simon \textsc{Gwerder}\\Tobias \textsc{Stauber}} % Author name

\date{\today} % Date for the report

\begin{document}
	
	\maketitle % Insert the title, author and date
	
	
	\section{Summary}
  Berschreibt für jede Entität einzeln, wie (lesend, schreibend, ausführend) auf welche Ressource zugegriffen werden darf.
	
	\section{Context}
	Einsatz bei sicherheitskritischen Ressourcen, auf die nicht jeder Benutzer oder Prozess beliebig Zugriff erhalten soll.
	
	\section{Problem}
	Aus Gründen der Vertraulichkeit oder Integrität. ist es manchmal wünschenswert, dass Ressourcen für bestimmte Benutzer oder Prozesse nicht frei verfügbar sind. Folgende Probleme müssen angegangen werden:\\
	\begin{itemize}{}{ }
		\item Die aktiven Benutzer und Prozess müssen Unterschieden werden können.
		\item Die Authorization muss unabhängig von den Ressourcen sein.
		\item Die Authorization sollte flexibel im Umgang mit Subjekten, Objekten und Rechten sein.
		\item Die Rechte von Benutzern und Prozessen sollen einfach modifiziert werden können.
	\end{itemize}

\section{Solution}
Jeder Subject (User, Prozess, usw.) zu Protection Object Verbindung ist ein Recht-Objekt zuweisen, welches ZugriffsTypen und ZugriffsRechte verwaltet und auch die Rechte beim Zugriff überprüft.

\begin{samepage}
\begin{verbatim}
 (User, Prozesse, ...)           (Resourcen wie File, DB, ...)
 .---------.                       .-------------------.
 | Subject |                       | Protection Object |
 |---------| * Authorization_rule *|-------------------|
 | id      |----------.------------| id                |
 |---------|          |            |-------------------|
 '---------'          |            '-------------------'
               .------'------.
               | Right       |
               |-------------|
               | access_type |
               | predicate   |
               | copy_flag   |
               |-------------|
               | checkRights |
               '-------------'

\end{verbatim}
\end{samepage}

\section{Consequences}
Es kann viel Aufwand kosten alle Rechte immer richtig zu vergeben. Dies ist vorallem bei grösseren Organisationen nicht zu unterschätzen. Zudem sind viele dieser Recht-Objekte zu verwalten und zu speichern.


\section{Known Uses}
Ein gutes Beispiel ist die Zugriffsrechtsverwaltung in einem Betriebssystem.
Oder auch in einer Datenbank oder einer Firewall. Jedoch sind zumeist Patterns eingesetzt, welche auf diesem Aufbauen.

\section{Relationships}
Das Pattern steht in einer engen Beziehung zum Role-Based Access Control Pattern.

\bibliographystyle{apalike}

\bibliography{sample}



\end{document}
