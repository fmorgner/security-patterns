\chapter{Reference Monitor}

Auch bekannt als POLICY ENFORCEMENT POINT

\section{Summary}
Dieses Pattern erzwingt deklarierte Zugriffseinschränkungen während dem Zugriff auf eine Ressource. Es beschreibt den abstrakten Prozess, der den Zugriff auf eine Ressource abfängt und ihn auf Einhaltung der Autorisierung überprüft.

\section{Context}
Rechenumgebung in welcher Benutzer oder Prozesse eine Anfrage für Daten oder Ressourcen machen.

\section{Problem}
Wenn Zugriffsbeschränkungen nicht durchgesetzt werden, dann ist es so, als würde es sie nicht geben. Prozesse oder Benutzer könnten alle Arten von unerlaubten Aktionen durchführen. Beispiel: Jeder Benutzer könnte jede Datei lesen.  Folgende Probleme müssen also abgedeckt werden:
\begin{itemize}
  \item Die Definition der Autorisierung ist nicht genug, sie muss bei Zugriffen von Prozessen oder Benutzern für jede Anfrage überprüft werden.
  \item Es gibt viele mögliche Wege den Zugriff zu beschränken, es braucht aber eine abstraktes Modell zur Authorisierungsdurchsetzung, das auf jede Ebene eines System anwendbar ist.
\end{itemize}

\section{Solution}
Es wird ein Prozess definiert, der den Zugriff auf Ressourcen abfängt und sie auf Einhaltung der Autorisierung überprüft.

\begin{figure}[H]
  \centering
  \includesvg[width=0.6\textwidth]{03-reference_monitor-class-diagram}
  \caption{Strukturdiagramm f\"ur Reference Monitor}
\end{figure}

\begin{figure}[H]
  \centering
  \includesvg[width=0.8\textwidth]{03-reference_monitor-sequence-diagram}
  \caption{Sequenzdiagramm f\"ur Reference Monitor}
\end{figure}

\section{Consequences}
\begin{itemize}
    \pro{Wenn alle Zugriffe abgefangen werden, kann sichergestellt werden, dass die Regeln erfüllt werden.}
    \pro{Implementierungen werden nicht durch den beschriebenen Prozess eingeschränkt}
    \con{Implementierungen sind für alle typen von Ressourcen nötig. Beispielsweise ein Dateimanager muss den Zugriff auf Dateien überprüfen.}
    \con{Jeden Zugriff zu kontrollieren kann das System drastisch ausbremsen. Es kann sinnvoll sein, die Überprüfung zur Compiler-Zeit zu machen, damit wird der wiederholte Zugriff zur Ausführungszeit nicht mehr ausgebremst.}
\end{itemize}

\section{Known Uses}
\begin{itemize}
  \item Die meisten Betriebsysteme (Bsp: Linux, MacOS X, Windows NT/2000/XP/Vista/8/9/10\ldots)
  \item Java Security Manager
  \item Datenbanksysteme
  \item Firewalls
\end{itemize}

\section{Relationships}
\begin{itemize}
  \item Reference Monitor ist ein Spezialfall von Check Point
  \item Implementierung: concrete Reference Monitor
  \item Interceptor (POSA2)
\end{itemize}

