\chapter{Identification and Authentication Requirements}

\section{Summary}
Ein "Identification and Authentication (I\&A) service" muss eine Auswahl an Anforderungen an den Service und an die Verfügbarkeit des Services erfüllen. Es ist die Aufgabe von I\&A Individuen zu erkennen und deren Identität zu bestätigen. Zwar ist jede Anfrage individuell, aber es existieren dennoch allgemeingültig Anforderungen.
Die "I\&A Requirements" bietet eine allgemeingültige Auswahl von Anforderungen und helfen die relative Wichtigkeit von konkurrierenden Anforderungen zu ermitteln.

\section{Context}
Eine Organisation oder ein Projekt, wo bewusst I\&A eingesetzt wird. Hier wird nur der Bereich behandelt, wo Identifikation und Authentifikation eingesetzt werden.

\section{Problem}
Anforderungen von I\&A stehen oft in Konflikt miteinander. Oft ist es nötig diese Anforderungen aufeinander abzustimmen. Beispielsweise stehen die Sicherheit des Systems der einfachen Benutzbarkeit gegenüber. Folgende Situationen sollten adressiert werden:
\begin{itemize}
  \item Besitzer von I\&A Services möchten, dass dieser korrekt funktioniert. Der Service soll erkennen ob ein Actor mit einem Identifier verbunden ist.
  \item Falsche Rückmeldung bei einem Betrugsfall, kann zu "Leaks" führen.
  \item Die Produktivität wird negativ beeinflusst, wenn ein ermächtigter Actor fälschlicherweise "ausgepserrt" wird.
  \item Benutzer verlangen I\&A Services mit guter Qualität: Schnelle Antwortzeiten, Korrekte Funktionalität, Einfach zu verstehen, Sicher, Passend zur Kategorie eines Benutzers
  \item Für Unternhehmungen sollen I\&A Services kosteneffektiv sein und einen gute ROI liefern.
  \item Der I\&A Service soll vor gestolenen Identitäten schützen.
\end{itemize}

\section{Solution}
Spezifizieren von I\&A Requirements für die vorliegende I\&A Domain. Dabei muss die Wichtigkeit jeder dieser Anforderung untersucht werden.

Die Lösung hat zwei Aspekte: Einen Anforderungs Prozess bzw. einen generellen Anforderungskatalog.

Zuerst wird die Domain festgelegt, für welche die I\&A Anforderungen spezifiziert werden. Daraus ergeben sich Faktoren, welche die Spezialisierung beeinflussen, sowie die Wichtigkeit der Anforderung definieren. Danach werden die Anforderungen spezifiziert.

\subsection{General Requirements}
\begin{itemize}
  \item Betrüger genau identifizieren
  \item Legitime Akteuere genau identifizieren
  \item Minimiere Unterschiede in den Usercharakteristiken
  \item Minimieren Zeitaufwand um den Service zu nutzen
  \item Minimiere das Benutzerrisiko
  \item Minimiere Benutzerkosten für die Installation
  \item Minimiere Veränderungen an existierenden System Infrastrukturen
  \item Minimiere Wartungs-, Management- und Overhead-Kosten
  \item Schütze den I\&A Service und seine Daten
  \item Unterschiede zwischen Requirements
\end{itemize}

\section{Consequences}
\begin{itemize}
    \pro{Erzwingt eine klare Definition der I\&A Domains, was das Verständniss der nötigen I\&A Domains und den konkreten Anforderungen fördert.}
    \pro{Führt zu klaren Begründungen für die Wahl der I\&A-Anforderungen.}
    \pro{Regt eine explizite Analyse der I\&A-Requirements an und führt so zu einer besseren Wahl der I\&A-Mechanismen}
    \pro{Es ergibt sich eine klare Dokumentation der I\&A-Anforderungen.}
    \con{Die Umsetzung benötigt Zeit und Personal. Es ist möglich, dass die Kosten die Nutzen überwiegen.}
    \con{Kann zu Overengineering führen. Ein weg das zu vermeiden ist nur die Teile des Patterns zu verwenden welche für die konkrete Anwendung den grössten Nutzen bringen.}
    \con{Spezialfälle können von den generischen I\&A-Requirements nicht abgedeckt werden. Hier empfiehlt es sich, selber neue Anforderungen zu spezifizieren.}
    \con{Die Anforderungen können sich über die Lebenszeit des Systems ändern, somit muss auch die Dokumentation aktualisierbar gestaltet sein.}
    \con{Verschiedene Mitarbeiter/Abteilungen haben verschiedene Auffassungen der Wichtigkeit der Anforderungen, was es schwierig macht die Anforderungen abzuwägen. Das kann auch ein Vorteil sein, da es intensievere Diskussionen und bewusstere Entscheidungen fördert.}
\end{itemize}

\section{Known Uses}
\begin{itemize}
  \item \textbf{OMB2003} Eine Regelung zur elektronischen Authentisierung von Personen welche (staatliche) Online Angebote nutzen.
  \item \textbf{ISO15408} Internationaler Standard der Evaluationskriterien für IT Sicherheit definiert.
  \item \textbf{Firesmith2003} Beschreibung funktionaler I\&A-Requirements und Diskussion von I\&A Domain im Zusammenhang mit diesen.
\end{itemize}

