\chapter{Integration Reverse Proxy}

\section{Problem}
\begin{itemize}
  \item Eine Gruppierung von Systemen, die jeweils aus bestimmten Gründen auf eigenen Maschinen laufen müssen.
  \item Diese Gruppierung sollte aus usability-Gründen trotz IP-Adressierung nicht direkt so dem Kunden dargeboten werden.
  \item Die Schnittstellen ywischen den einzelnen Maschinen soll sich nie ändern, auch nicht, wenn die Maschinen neu konfiguriert werden.
  \item Falls einzelne Instanzen der Systeme nicht mehr ausreichen, soll eine, die Last verteilende, Lösung möglich sein.
  \item Wenn gleichzeitig noch Protection Reverse Proxy eingesetzt wird, sollte es möglich sein nur ein einzelnes SSL Certificat zu verwenden, statt für jede Maschine eines.
  \item Ausserdem sollen Backend Schnittstellen zwischen einzelnen Applikationshosts sich nicht ändern, auch wenn die dahinter liegenden Hosts rekonfiguriert werden. Dies stellt die Flexibilität bei Änderungen am Backend sicher.
\end{itemize}

\section{Solution}
\begin{itemize}
  \item Implementiere den IRP, welcher auf der Internet-Seite eine IP-Adresse/URL aufweist.
  \item Mappe alle URLs der Maschinen auf IP-Adressen/URL's und hinterlege dieses Mapping beim IRP.
  \item Es kann auch auf dem IRP eine Liste der verfügbaren Maschinen/Services hinterlegt werden, welche bei direkter Anwahl angeboten wird.
  \item Ändert nun die Zuständigkeit eines Servers, kann diese Änderung zentral am IRP vorgenommen werden.
  \item Das SSL-Cert kann nun zentral beim IRP hinterlegt werden.
\end{itemize}
\begin{figure}[H]
  \centering
  \includesvg[width=0.5\textwidth]{29-integration_reverse_proxy-sequence-diagram}
  \caption{Sequenzdiagramm f\"ur CHAPTER}
\end{figure}
\subsection{Implementation}
Die Implementierung ist die des Protection-Reverse-Proxys, mit folgenden Erweiterungen:
\begin{itemize}
  \item Es gibt nur eine URL, welche als Einstiegspunkt dient z.B. 'http://epicnonexistentwebshop.ch/'.
  \item Die Gruppierten Systeme so konfigurieren, dass sie nur vom Proxy her verfügbar sind.
  \item Optional: Failover Proxy konfigurieren so, dass dieser bei einem Versagen des Hauptproxys übernehmen kann.
  \item Optional: Load-Balancing um die Last des Proxys zu verteilen.
\end{itemize}

\section{Consequences}
\begin{itemize}
    \pro{Nur der Proxy ist direkt dem Internet und damit potentiellen Angriffen ausgesetzt}
    \pro{Die Netzwerktopologie der Service-Maschinen ist vom Client verborgen und kann also flexibler gehandhabt werden.}
    \pro{Failover-Hosts und Load-Balancing sind möglich.}
    \pro{Eingehende Requests auf die Services können vom Proxy zentral geloggt werden.}
    \pro{Authentisierung kann auf dem Proxy zentral gelöst werden}
    \pro{Es muss nur ein Certificat bezahlt werden, welches von einer akzeptierten CA ausgestellt wurde.}
    \con{Falls auf Failover-Hosts verzichtet wir, ist der Proxy ein Sinlge-Point-of-Failure.}
    \con{Session handling und Testing sind komplizierter}
    \con{Bei grossen Systemen mit vielen Requests wird es schwierig all diese mit nur einem Host zu bearbeiten.}
\end{itemize}

